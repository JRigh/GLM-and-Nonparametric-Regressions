\documentclass[8p, a4paper] {article}

\usepackage{geometry}
 \geometry{
 a4paper,
 total={130mm,300mm},
 left=40mm,
 top=25mm,
 }

\usepackage [T1] {fontenc}
\usepackage{lmodern}
\usepackage[utf8] {inputenc}
\usepackage[english] {label}
\usepackage{amssymb}
\usepackage{amsmath}
\usepackage[round] {natbib}
\usepackage[pdftex]{graphicx}
\usepackage{graphicx}
\usepackage[11pt]{moresize}
\usepackage{listings}
\usepackage[svgnames]{xcolor}
\graphicspath{  {C:/Users/Righ/Desktop/images/}  }
\usepackage{color}     
\usepackage{lmodern}
\def\one{\mbox{1\hspace{-4.25pt}\fontsize{12}{14.4}\selectfont\textrm{1}}} % 11pt 

\newenvironment{spmatrix}[1]
 {\def\mysubscript{#1}\mathop\bgroup\begin{pmatrix}}
 {\end{pmatrix}\egroup_{\textstyle\mathstrut\mysubscript}}

\lstset{language=R,
    stringstyle=\color{DarkGreen},
    otherkeywords={0,1,2,3,4,5,6,7,8,9},
    morekeywords={TRUE,FALSE},
    deletekeywords={data,frame,length,as,character},
    keywordstyle=\color{blue},
    commentstyle=\color{teal},
}



\author{Julian Sampedro}
\title{}
\date{\today}



\begin{document}

\noindent
\begin{center}
\textbf{\huge Question 1}
\end{center}


\vspace{15}
\noindent
Q: Download the dataset, and plot it in R. Upload a picture of your graph displaying the data and comment on the features of the data. Does it present any trends or quasi-periodic behavior? 


\begin{figure}[h!]
\begin{center}
\includegraphics[width=14cm]{plot1R}
\end{center}
\end{figure}


A: The data, from the plot, does not exhibit any particular linear trend or periodic behavior and appear to be stationary: constant variance and mean over time.


\vspace{15}
\noindent
Code:

\tiny
\begin{lstlisting}[language=R]
# import data
yt = read.delim("C:/Users/julia/OneDrive/Desktop/Statistics notes/w50y2024/data.txt", 
header = FALSE)
yt = as.numeric(data$V1)
plot(yt, type = 'l', col = "blue", main = 'Plot of data',
     xlab = "Index", ylab = "Value", 
     main = "Plot of Data")
\end{lstlisting}



\normalsize

\vspace{15}
\noindent
\begin{center}
\textbf{\huge Question 2}
\end{center}

\vspace{15}
\noindent
Q: Modify the code below to obtain the maximum likelihood estimators (MLEs) for the AR coefficients under the conditional likelihood. For this you will assume an autoregressive model of order $p=8$. The parameters of the model are $\boldsymbol{\phi} = ( \phi_{1},... , \phi_{8})^{T}$ and $v$. You will compute the MLE of $\boldsymbol{\phi}$, denoted as $\hat{\boldsymbol{\phi}}.

\vspace{15}
\noindent
A: The maximum likelihood estimates are $\hatä\boldsymbol{\phi}} = ( 1.60987, -0.8934844, -0.0004550485, 0.008435584, \\
-0.02283475, 0.001826861, -0.01358144, 0.00909051)^{T}$.

\newpage
\noindent
Code:


\tiny
\begin{lstlisting}[language=R]
## Case 1: Conditional likelihood
set.seed(2024)
p=8
y=rev(yt[(p+1):T]) # response
X=t(matrix(yt[rev(rep((1:p),T-p)+rep((0:(T-p-1)),rep(p,T-p)))],p,T-p));
XtX=t(X)%*%X
XtX_inv=solve(XtX)
phi_MLE=XtX_inv%*%t(X)%*%y # MLE for phi
s2=sum((y - X%*%phi_MLE)^2)/(length(y) - p) #unbiased estimate for v

cat("\n MLE of conditional likelihood for phi: ", phi_MLE, "\n",
    "Estimate for v: ", s2, "\n")
 MLE of conditional likelihood for phi:  1.60987 -0.8934844 -0.0004550485 0.008435584 -0.02283475 0.001826861 
-0.01358144 0.00909051 
 Estimate for v:  0.9568277 
\end{lstlisting}

\normalsize

\noindent
\begin{center}
\textbf{\huge Question 3}
\end{center}


\vspace{15}
\noindent
Q: Obtain an unbiased estimator for the observational variance of the $AR(8)$.. You will compute the unbiased estimator for $v$ denoted as $s^{2}$.

\vspace{15}
\noindent
A: The estimator for $v$ is and $s^{2} = 0.9568277$.


\normalsize

\noindent
\begin{center}
\textbf{\huge Question 4}
\end{center}


\vspace{15}
\noindent
Q: Modify the code below to obtain $500$ samples from the posterior distribution of the parameters  $\boldsymbol{\phi} = ( \phi_{1},... , \phi_{8})^{T}$ and $v$. Once you obtain samples from the posterior distribution you will compute the posterior means of $\boldsymbol{\phi}$ and $v$, denoted as $\hat{\boldsymbol{\phi}}$​ and $\hat{v}$ respectively.

\vspace{15}
\noindent
A: The posterior means are $\hat{\boldsymbol{\phi}} = ( 1.605355159, -0.885135355, -0.004239663,  0.007291180, \\
 -0.026931834,  0.011587524, -0.021160654,  0.011207555)^{T}$. The rounded results are $\hat{\boldsymbol{\phi}} = (1.605, -0.885, -0.004,  0.007, -0.027,  0.012, -0.021,  0.011)^{T}$ and $\hat{v} = 0.967$ .


\vspace{15}
\noindent
Code:

\tiny
\begin{lstlisting}[language=R]
n_sample=500 # posterior sample size
library(MASS)

## step 1: sample v from inverse gamma distribution
v_sample=1/rgamma(n_sample, (T-2*p)/2, sum((y-X%*%phi_MLE)^2)/2)

## step 2: sample phi conditional on v from normal distribution
phi_sample=matrix(0, nrow = n_sample, ncol = p)
for(i in 1:n_sample){
  phi_sample[i, ]=mvrnorm(1,phi_MLE,Sigma=v_sample[i]*XtX_inv)
}

#posterior means
apply(phi_sample, 2, mean)
# [1]  1.605355159 -0.885135355 -0.004239663  0.007291180 
# -0.026931834  0.011587524 -0.021160654  0.011207555

round(apply(phi_sample, 2, mean),3)

#rounded variance estimate
round(mean(v_sample), 3)
\end{lstlisting}



\normalsize

\newpage

\noindent
\begin{center}
\textbf{\huge Question 5}
\end{center}


\vspace{15}
\noindent
Q: Modify the code below to use the function polyroot and obtain the moduli and periods of the reciprocal roots of the AR polynomial evaluated at the posterior mean $\hat{\boldsymbol{\phi}}$.

\vspace{15}
\noindent
A: The moduli and periods of the reciprocal roots would then be: 


\vspace{5}
\noindent
Moduli: 0.963, 0.472, 0.963, 0.506, 0.506, 0.495, 0.472, 0.428


\vspace{5}
\noindent
Periods: -1.179700e+01, -3.019000e+00,  1.179700e+01,  5.800000e+00, -5.800000e+00,  2.000000e+00,  3.019000e+00, -1.580598e+16



\vspace{15}
\noindent
Code:

\tiny
\begin{lstlisting}[language=R]
# Assume the folloing AR coefficients for an AR(8)
phi= apply(phi_sample, 2, mean)
roots=1/polyroot(c(1, -phi)) # compute reciprocal characteristic roots
r=Mod(roots) # compute moduli of reciprocal roots
lambda=2*pi/Arg(roots) # compute periods of reciprocal roots
# print results modulus and frequency by decreasing order
print(cbind(r, abs(lambda))[order(r, decreasing=TRUE), ][c(2,4,6,8),]) 

#                         r             
# [1,] 0.9632800 1.179690e+01
# [2,] 0.5057569 5.799588e+00
# [3,] 0.4720255 3.019354e+00
# [4,] 0.4280434 1.580598e+16
\end{lstlisting}



\normalsize


\end{document}